\documentclass[xcolor={dvipsnames,table}]{beamer}
\usepackage{epsfig,graphicx}
\usepackage{palatino}
\usepackage{fancybox}
\usepackage{relsize}
\usepackage[procnames]{listings}
\usepackage{hyperref}
\usepackage{qtree} % needed?
\usepackage{booktabs}
\usepackage{dirtree}
\usepackage[normalem]{ulem}


% fatter TT font
\renewcommand*\ttdefault{txtt}
% another TT, suggested by Alex
% \usepackage{inconsolata}
% \usepackage[T1]{fontenc} % needed as well?


\newcommand{\scale}{0.7}

\newcommand{\todo}[1]{{\emph{TODO: #1}}}
\newcommand{\martin}[1]{{\color{blue} Martin: #1}}
\newcommand{\abcdef}[1]{{\color{red} Author2: #1}}

% uncomment following for final submission
%\renewcommand{\todo}[1]{}
%\renewcommand{\martin}[1]{}
%\renewcommand{\author2}[1]{}

\newcommand{\code}[1]{{\texttt{#1}}}

\hypersetup{
  linkcolor  = black,
%  citecolor  = blue,
  urlcolor   = blue,
  colorlinks = true,
}

\beamertemplatenavigationsymbolsempty
\setbeamertemplate{footline}[frame number]





\newif\ifbook
\input{../shared/chisel}

\title{Testing and Verification}
\author{Martin Schoeberl}
\date{\today}
\institute{Technical University of Denmark\\
Embedded Systems Engineering}

\begin{document}

\begin{frame}
\titlepage
\end{frame}

%\begin{frame}[fragile]{TODO 2024, this lecture and later ones}
%\begin{itemize}
%\item This is material for barely one hour -- extend it
%\item Maybe a better test example. At least make it a 4 Mux and write when it is done, same for the Heap example
%\item Talk about directions and what is left and what is right
%\item Maybe Chisel/VHLD comparison
%\end{itemize}
%\end{frame}

\begin{frame}[fragile]{Overview}
\begin{itemize}
\item Review components
\item A little bit of Scala (for testing)
\item Debugging and testing
\item Digital designers (sometimes) call testing verification
\begin{itemize}
\item To distinguish from final chip testing
\end{itemize}
\end{itemize}
\end{frame}

\begin{frame}[fragile]{DTU Chip Day}
\begin{itemize}
\item Note the date: Tu 16 April afternoon
\item Start with sandwiches and finish with beer
\item Presentation of chip design and verification work/companies in Denmark
\item Several chip companies will present and are participating
\item Opportunity to network for: theses with companies, internship, student jobs
\end{itemize}
\end{frame}

%\begin{frame}[fragile]{Some Clarification - TODO: Maybe drop or change}
%\begin{itemize}
%\item Online learning is hard, I understand
%\item Installing all tools on your laptop can also be hard
%\begin{itemize}
%\item We are back physical now
%\end{itemize}
%\item Don't be afraid from the \emph{power user} demos
%\begin{itemize}
%\item Just for ``entertainment'' and for advanced users
%\item You can also simply ignore it
%\end{itemize}
%\item Learning Objectives
%\begin{itemize}
%\item How to build medium sized digital circuits
%\item Describe them in Chisel  and implement them in an FPGA
%\item Timing analysis of digital circuits
%\end{itemize}
%\end{itemize}
%\end{frame}

\begin{frame}[fragile]{Direction of a Connection}
\begin{itemize}
\item The flow on a \code{Wire} has a direction
\item One end is the output/driver/source and the other end is the input/sink
\item Or producer and consumer
\item An expression has also a direction:
\begin{itemize}
\item The right hand site produces a value
\item The left hand site consumes the value
\end{itemize}
\item \code{sink := source1 + source2}
\item Like in Java and other programming languages
\item Draw a figure
\end{itemize}
\end{frame}

\begin{frame}[fragile]{Last Chisel Lab (week 3)}
\begin{itemize}
\item On components and small sequential circuits
\begin{itemize}
\item Registers plus combinational circuits
\end{itemize}
\item Did you finish the exercises?
\begin{itemize}
\item Do the poll
\end{itemize}
\item They are not mandatory, but helpful for preparation for the final project
\item Let's look at solutions
\end{itemize}
\end{frame}

\begin{frame}[fragile]{Components are Modules}
\begin{itemize}
\item Components are building blocks
\begin{itemize}
\item Like concrete, physical ICs
\end{itemize}
\item Components have input and output ports (= pins)
\begin{itemize}
\item Organized as a \code{Bundle}
\item Assigned to the field \code{io}
\end{itemize}
\item We build circuits as a hierarchy of components
\begin{itemize}
\item You did a 4:1 multiplexer out of three 2:1 multiplexers
\end{itemize}
\item In Chisel a component is called \code{Module}
\item Components/Modules are used to organize the circuit
\begin{itemize}
\item Similar to using methods in Java
\item But they are connected with \emph{wires}
\end{itemize}
\end{itemize}
\end{frame}


\begin{frame}[fragile]{A Binary Watch}
\begin{itemize}
\item Built out of discrete, digital components
\end{itemize}
\begin{figure}
    \centering
    \href{https://commons.wikimedia.org/wiki/File:Relogio_binario.JPG}{\includegraphics[scale=0.3]{binary-watch.jpg}}
\end{figure}

{\tiny Source: Diogo Sousa, \href{https://en.wikipedia.org/wiki/Public_domain}{public domain}}
\end{frame}


\begin{frame}[fragile]{Let Us Build a Counter}
\begin{itemize}
\item Counting from 0 up to 9
\item Restart from 0
\item Build it out of components
\item We need:
\begin{itemize}
\item Adder
\item Register
\item Multiplexer
\end{itemize}
\item
\item But these are very tiny components
\end{itemize}
\end{frame}

\begin{frame}[fragile]{An Adder (Component/Module)}
\begin{columns}
\begin{column}{0.4\textwidth}
\begin{figure}
  \includegraphics[scale=\scale]{../figures/components-adder}
\end{figure}
\end{column}
\begin{column}{0.6\textwidth}
\shortlist{../code/components_add.txt}
\end{column}
\end{columns}
\end{frame}

\begin{frame}[fragile]{A Register}
\begin{columns}
\begin{column}{0.4\textwidth}
\begin{figure}
  \includegraphics[scale=\scale]{../figures/components-register}
\end{figure}
\end{column}
\begin{column}{0.6\textwidth}
\shortlist{../code/components_reg.txt}
\end{column}
\end{columns}
\end{frame}

\begin{frame}[fragile]{The Counter Schematics}
\begin{figure}
  \includegraphics[scale=0.6]{../figures/components-counter}
\end{figure}
\end{frame}

\begin{frame}[fragile]{The Counter in Chisel}
\shortlist{../code/components_cnt.txt}
\end{frame}

\begin{frame}[fragile]{Summarize Components}
\begin{itemize}
\item Think like concrete components (ICs)
\item They have named pins (\code{io.name})
\begin{itemize}
\item In hardware language these pins are often called ports
\item Ports have a direction (input or output)
\end{itemize}
\item They need to be created:
\begin{itemize}
\item \code{val mc = Module(new MyComponent())}
\end{itemize}
\item and pins need to be connected with \code{:=}
\item One module is special, as it is the top module
\end{itemize}
\end{frame}

%\begin{frame}[fragile]{XXX}
%\begin{itemize}
%\item xxx
%\item xxx
%\item xxx
%\item xxx
%\item xxx
%\end{itemize}
%\end{frame}



\begin{frame}[fragile]{Chisel Main}

\begin{itemize}
\item Create one top-level Module
\item Invoke the \code{emitVerilog()} from the App
\item Pass the top module (e.g., \code{new Hello()})
\item Optional: pass some parameters (in an \code{Array})
\item Following code generates Verilog code for the \emph{Hello World}
\end{itemize}
\shortlist{../code/generate.txt}
\end{frame}

\begin{frame}[fragile]{Scala}
\begin{itemize}
\item Is object oriented
\item Is functional
\item Strongly typed with very good type inference
\item Runs on the Java virtual machine
\item Can call Java libraries
\item Consider it as Java++
\begin{itemize}
\item Can almost be written like Java
\item With a more lightweight syntax
% \item Scala for Java Refugees is a nice tutorial (but link is dead)
% \item \url{http://www.codecommit.com/blog/scala/roundup-scala-for-java-refugees}
\end{itemize}
\end{itemize}
\end{frame}

\begin{frame}[fragile]{Scala Hello World}
\shortlist{../src/main/scala/HelloScala.scala}
\begin{itemize}
\item Compile with \code{scalac} and run with \code{scala}
\item You can even use Scala as a scripting language
\item Or run with \code{sbt run}
\item Show both
\end{itemize}
\end{frame}

\begin{frame}[fragile]{Scala Values and Variables}
\begin{itemize}
\item Scala has two type of variables: \code{val}s and \code{var}s
\item A \code{val} cannot be reassigned, it is a constant
\item We use a \code{val} to name a hardware component in Chisel
\end{itemize}
\shortlist{../code/scala_val.txt}
\begin{itemize}
\item Types are usually inferred
\item But can be explicitly stated as follows
\end{itemize}
\shortlist{../code/scala_int_type.txt}
\end{frame}

\begin{frame}[fragile]{Scala Variables}
\begin{itemize}
\item A \code{var} can be reassigned, it is like a classic variable
\item We use a \code{var} to write a hardware generator in Chisel
\end{itemize}
\shortlist{../code/scala_var.txt}
\end{frame}

\begin{frame}[fragile]{Simple Loops}
\shortlist{../code/scala_loop.txt}
\begin{itemize}
\item We can use a loop for testing
\end{itemize}
\end{frame}

\begin{frame}[fragile]{Scala \code{for} Loop for Circuit Generation}
\shortlist{../code/scala_loop_gen.txt}
\begin{itemize}
\item \code{for} is Scala
\item This loop generates several connections
\item The connections are parallel hardware
\item This is a shift register
\end{itemize}
\end{frame}

\begin{frame}[fragile]{Conditions}
\shortlist{../code/scala_condition.txt}
\begin{itemize}
\item Executed at runtime, when the circuit is created
\item This is \emph{not} a mlutplexer
\end{itemize}
\end{frame}


\begin{frame}[fragile]{Testing and Debugging}
\begin{itemize}
\item Nobody writes perfect code ;-)
\item We need a method to improve the code
\item In Java we can simply print values:
\begin{itemize}
\item \code{println("42");}
\end{itemize}
\item What can we do in hardware?
\begin{itemize}
\item Describe the whole circuit and hope it works?
\item We can switch an LED on and off
\item Test it with switches and LEDs in an FPGA
\end{itemize}
\item We need some tools for \href{https://en.wikipedia.org/wiki/Debugging#/media/File:H96566k.jpg}{debugging}
\item Writing testers in Chisel
\item We test by running a simulation of the circuit
\end{itemize}
\end{frame}

\begin{frame}[fragile]{ScalaTest}
\begin{itemize}
\item Testing framework for Scala and Java
\item Tests are placed under \code{src/test/scala}
\item \code{sbt} understands ScalaTest
\item Run all tests with:
\begin{chisel}
sbt test
\end{chisel}
\item When all (unit) tests are ok, the test suit passes
\item A little bit funny syntax
\item ChiselTest is based on ScalaTest
\end{itemize}
\end{frame}

\begin{frame}[fragile]{Testing with Chisel}
\begin{itemize}
\item A test contains
\begin{itemize}
\item a device under test (DUT) and
\item the testing logic
\end{itemize}
\item Set input values with \code{poke}
\item Advance the simulation with \code{step}
\item Read the output values with \code{peekInt}
\item Compare the values with \code{expect}
\item Import following packages
\shortlist{../code/test_import.txt}
\end{itemize}
\end{frame}

\begin{frame}[fragile]{An Example DUT}
\begin{itemize}
\item A device-under test (DUT)
\item Just 2-bit AND logic and equvicalence
\shortlist{../code/test_dut.txt}
\end{itemize}
\end{frame}

\begin{frame}[fragile]{A ChiselTest}
\begin{itemize}
\item Extends class \code{AnyFlatSpec} with \code{ChiselScalatestTester}
\item Has the device-under test (DUT) as parameter of the \code{test()} function
\item Test function contains the test code
\item Testing code can use all features of Scala
\item Is placed in \code{src/test/scala}
\item Is run with \code{sbt test}
\end{itemize}
\end{frame}

\begin{frame}[fragile]{A Simple Tester}
\begin{itemize}
\item Just using \code{println} for manual inspection
\shortlist{../code/test_bench_simple.txt}
\end{itemize}
\end{frame}


\begin{frame}[fragile]{A Real Tester}
\begin{itemize}
\item Poke values and \code{expect} some output
\shortlist{../code/test_bench.txt}
\end{itemize}
\end{frame}



\begin{frame}[fragile]{Generating Waveforms}
\begin{itemize}
\item Waveforms are timing diagrams
\item Good to see many parallel signals and registers
\begin{verbatim}
sbt "testOnly SimpleTest -- -DwriteVcd=1"
\end{verbatim}
\item Or setting an attribute for the \code{test()} function
\begin{chisel}
test(new DeviceUnderTest)
    .withAnnotations(Seq(WriteVcdAnnotation))
\end{chisel}
\item IO signals and registers are dumped
\item Option \code{--debug} puts all wires into the dump
\item Generates a .vcd file in
\begin{chisel}
    test_run_dir/test-name
\end{chisel}
\item Viewing with GTKWave or ModelSim
\end{itemize}
\end{frame}

\begin{frame}[fragile]{Display Waveform with GTKWave}
\begin{itemize}
\item Run the tester: \code{sbt test}
\item Locate the .vcd file in test\_run\_dir/...
\item Start GTKWave
\item Open the .vcd file with
\begin{itemize}
\item File -- Open New Tab
\end{itemize}
\item Select the circuit
\item Drag and drop the interesting signals
\end{itemize}
\end{frame}

\begin{frame}[fragile]{Waveform Testing Demo}
\begin{itemize}
\item Counter with a limit from last Chisel lab (\code{Count6})
\item Show Count6 tester: the original and the waveform
\item Run it and look at waveform
\item Add the solution
\item Run again and reload the waveform
\end{itemize}
\end{frame}

\begin{frame}[fragile]{A Self-Running Circuit}
\begin{itemize}
\item \code{Count6} is a self-running circuit
\item Needs no stimuli (\code{poke})
\item Just run for a few cycles
\end{itemize}
\begin{chisel}
    test(new Count6) { dut =>
      dut.clock.step(20)
    }
\end{chisel}
\end{frame}

\begin{frame}[fragile]{The WaveForm}
\begin{itemize}
\item The complete test
\item Note the \code{.withAnnotations(Seq(WriteVcdAnnotation)}
\end{itemize}
\begin{chisel}
class Count6WaveSpec extends AnyFlatSpec with ChiselScalatestTester {
  "CountWave6 " should "pass" in {
    test(new Count6).withAnnotations(Seq(WriteVcdAnnotation)) { dut =>
      dut.clock.step(20)
    }
  }
}
\end{chisel}
\end{frame}

\begin{frame}[fragile]{Display Waveform with GTKWave}
\begin{itemize}
\item Run the tester: \code{sbt test}
\item Locate the .vcd file in test\_run\_dir/...
\item Start GTKWave
\item Open the .vcd file with
\begin{itemize}
\item File -- Open New Tab
\end{itemize}
\item Select the circuit
\item Drag and drop the interesting signals
\end{itemize}
\end{frame}

\begin{frame}[fragile]{Vending Machine Testing}
\begin{itemize}
\item I provide a minimal tester to generate a waveform
\item Adding some coins and buying
\item You can and shall extend this tester
\item Better having more than one tester
\item Show the waveform of the test
\end{itemize}
\end{frame}


\begin{frame}[fragile]{Printf Debugging}
\begin{itemize}
\item We can \emph{print} in the hardware during simulation
\item Printing happens on the rising edge of the clock
\item Good to see many parallel signals and registers
\item \code{printf} anywhere in the module definition
\end{itemize}
\shortlist{../code/test_dut_printf.txt}
\end{frame}



\begin{frame}[fragile]{Test Driven Development (TDD)}
\begin{itemize}
\item Software development process
\begin{itemize}
\item Can we learn from SW development for HW design?
\end{itemize}
\item Writing the test first, then the implementation
\item Started with extreme programming
\begin{itemize}
\item Frequent releases
\item Accept change as part of the development
\end{itemize}
\item A path to \emph{Agile Hardware Development!}
\item Not used in its pour form
\begin{itemize}
\item Writing all those tests is simply considerer too much work
\item \textbf{But}, write at least one test for each component
\end{itemize}
\end{itemize}
\end{frame}

\begin{frame}[fragile]{Regression Tests}
\begin{itemize}
\item Tests are collected over time
\item When a bug is found, a test is written to reproduce this bug
\item Collection of tests increases
\item Runs every night to test for \emph{regression}
\begin{itemize}
\item Did a code change introduce a bug in the current code base?
\end{itemize}
\end{itemize}
\end{frame}


\begin{frame}[fragile]{Continuous Integration (CI)}
\begin{itemize}
\item Next logical step from regression tests
\item Run all tests whenever code is changed
\item Automate this with a repository, e.g., on GitHub
\item Run CI on GitHub
\item Show about this on the Chisel book
\begin{itemize}
\item Show \code{sbt test}
\item Live demo on GitHub
\item Mail from GitHub when it fails
\end{itemize}
\item \url{https://github.com/schoeberl/chisel-book/actions}
\item Maybe show how to set this up (it is easy ;)
\begin{itemize}
\item Start with the \href{https://github.com/schoeberl/chisel-empty}{chisel-empty} template
\item Open it with IntelliJ
\item Add action in GitHub
\end{itemize}
\end{itemize}
\end{frame}

\begin{frame}[fragile]{Testing versus Debugging}
\begin{itemize}
\item Debugging is during code development
\item Waveform and \code{println} are easy tools for debugging
\item Debugging does not help for regression tests
\item Write small test cases for regression tests
\item Keeps your code base \emph{intact} when doing changes
\item Better confidence in changes not introducing new bugs
\end{itemize}
\end{frame}


\begin{frame}[fragile]{Scala Build Tool (sbt)}
\begin{itemize}
\item Downloads Scala compiler if needed
\item Downloads dependent libraries (e.g., Chisel)
\item Compiles Scala programs
\item Executes Scala programs
\item Does a lot of magic, maybe too much
\item Compile and run with:
\end{itemize}
\begin{chisel}
sbt "runMain simple.Example"
sbt run
sbt test
sbt "testOnly MySpec"
sbt compile
\end{chisel}
\end{frame}

\begin{frame}[fragile]{Build Configuration}
\begin{itemize}
\item File name: \code{build.sbt}
\item Defines needed Scala version
\item Library dependencies
\end{itemize}
\begin{chisel}
scalaVersion := "2.12.13"

scalacOptions ++= Seq("-feature", "-language:reflectiveCalls")

resolvers ++= Seq(Resolver.sonatypeRepo("releases"))

addCompilerPlugin("edu.berkeley.cs" % "chisel3-plugin" % "3.5.0" cross CrossVersion.full)
libraryDependencies += "edu.berkeley.cs" %% "chisel3" % "3.5.0"
libraryDependencies += "edu.berkeley.cs" %% "chiseltest" % "0.5.0"
\end{chisel}
\end{frame}

\begin{frame}[fragile]{Today's Lab}
\begin{itemize}
\item Testing a faulty multiplexer
\item Do not look into the multiplexer code, find out with testing
\item Use ChiselTest (the description has been updated)
\item You have to start from scratch with the tester
\item Show and discuss your testing code with a TA (or me)
\item \href{https://github.com/schoeberl/chisel-lab/tree/master/lab4}{Lab 4}
\item And an additional challenge brought to you by Tjark
\end{itemize}
\end{frame}

\begin{frame}[fragile]{Summary}
\begin{itemize}
\item Small sequential circuits are our building blocks
\item We build larger circuits by combining components (modules)
\item There is no \emph{println} in (real) hardware
\item We need to write tests for the development
\item Debugging versus regression tests
\end{itemize}
\end{frame}

\end{document}

